\documentclass[UTF8,a4paper]{paper}
\usepackage{ctex}
\usepackage[utf8]{inputenc}
\usepackage{amsmath}
\usepackage{amssymb}
\usepackage{amstext} 
\usepackage{pdfpages}
\usepackage{graphicx}
\usepackage{wrapfig}
\usepackage{listings}
\usepackage{multicol}
\usepackage{float}
\newcommand{\tabincell}[2]{\begin{tabular}{@{}#1@{}}#2\end{tabular}}
\title{第二次书面作业}
\author{张蔚桐\ 2015011493\ 自55}
\begin {document}
\maketitle
\section{}
\subsection{}
列出真值表
\begin{table}[H]
\centering
\caption{(1)}
\begin{tabular}{|c|c|c|c|c|}
\hline
$P$&$Q$&$\neg P\lor\neg Q$&$P\leftrightarrow \neg Q$&$(\neg P\lor\neg Q)\rightarrow(P\leftrightarrow \neg Q)$\\
\hline 
T&T&F&F&T\\
\hline
T&F&T&T&T\\
\hline
F&T&T&T&T\\
\hline
F&F&T&F&F\\
\hline
\end{tabular}
\end{table}
因此得到析取范式为$P\lor Q$
合取范式为$P\lor Q$

\subsection{}
这个式子本身是析取范式,于是他的析取范式就是它本身
$$(P\land \neg Q\land S)\lor(\neg P\land Q\land R)$$

对这个式子取反,得到$$(\neg P\land\neg Q)\lor(\neg P\land Q\land\neg R)\lor(P\land Q\land\neg S)\lor(P\land\neg Q\land S)\lor(P\land Q\land S\land R)$$
再次取反得到合取范式

$$(P\lor Q)\land(P\lor\neg Q\lor R)\land(\neg P\lor\neg Q\lor S)\land(\neg P\lor Q\lor\neg S)\land(\neg P\lor\neg Q\lor\neg S\lor\neg R)$$
\section{}
\begin{enumerate}
\item 对于所有的$x$存在$y$使得$x+y=0$\ \ \ \ 正确
\item 存在一个$x$对于所有$y$使得$x+y=0$\ \ \ \ 错误
\end{enumerate}
\section{}
\subsection{}
写出真值表
\begin{table}[H]
\centering
\caption{(1)}
\begin{tabular}{|c|c|c|c|c|c|c|}
\hline
$P$&$Q$&$R$&$P\lor\neg Q$&$(P\lor\neg Q)\rightarrow R$&$P\land R$&$((P\lor\neg Q)\rightarrow R)\rightarrow (P\land R)$\\
\hline
T&T&T&T&T&T&T\\
\hline
T&T&F&T&F&F&T\\
\hline
T&F&T&T&T&T&T\\
\hline
T&F&F&T&F&F&T\\
\hline
F&T&T&F&T&F&F\\
\hline
F&T&F&F&T&F&F\\
\hline
F&F&T&T&T&F&F\\
\hline
F&F&F&T&F&F&T\\
\hline
\end{tabular}
\end{table}
因此可以得到合取范式
$(P\lor\neg Q)\land(P\lor Q\lor\neg R)$

因此得到子句集为$\{(P\lor\neg Q),(P\lor Q\lor\neg R)\}$
\subsection{}
\begin{equation}\begin{aligned}
G=&(\forall x)\{(\neg P(x)\lor(\forall y [\neg P(y)\lor P(f(x,y))]))\land(\neg(\forall y)[\neg Q(x,y)\lor P(y)])\}\\
=&(\forall x)\{(\neg P(x)\lor(\forall y [\neg P(y)\lor P(f(x,y))]))\land(\neg(\forall z)[\neg Q(x,z)\lor P(z)])\}\\
=&(\forall x)\{(\neg P(x)\lor(\forall y [\neg P(y)\lor P(f(x,y))]))\land((\exists z)[Q(x,z)\land\neg P(z)])\}\\
=&(\forall x\forall y\exists z)\{(\neg P(x)\lor\neg P(y)\lor P(f(x,y)))\land Q(x,z)\land\neg P(z)\}\\
\end{aligned}\end{equation}
得到子句集$\{(\neg P(x)\lor\neg P(y)\lor P(f(x,y))), Q(x,z),\neg P(g(x,y))\}$
\section{}
\begin{enumerate}
\item 存在,$\sigma = \{a/x,b/y,b/z\}$
\item 存在,$\sigma = \{g(f(v),g(u))/x\}$
\item 如果存在$f(x)=x$的解$x=x_0$则存在,$\sigma = \{x_0/x,x_0/y\}$,否则不存在
\item 存在,$\sigma = \{b/x,b/y,b/z\}$
\end{enumerate}
\section{}
\subsection{}
\begin{equation}\begin{aligned}
G=&\neg(\exists xP(x)\lor\exists yQ(y))\lor(\exists z(P(z)\lor Q(z)))\\
=&(\forall x\neg P(x) \land\forall y\neg P(y))\lor(\exists z(P(z)\lor Q(z)))\\
=&(\forall x\forall y\exists z)(\neg P(x) \land\neg P(y))\lor P(z)\lor Q(z)\\
=&(\forall x\forall y\exists z)((\neg P(x)\lor P(z)\lor Q(z))\land(\neg P(y)\lor P(z)\lor(Q(z)))\\
=&(\neg P(x)\lor P(g(x,y))\lor Q(z))\land(\neg P(y)\lor P(g(x,y))\lor(Q(g(x,y)))
\end{aligned}\end{equation}
\subsection{}
\begin{equation}\begin{aligned}
G=&(\forall x)(\neg P(x)\lor((\forall y)(\neg(\forall z)Q(z,y)\lor\neg(\forall z)(R(y,z)))\\
=&(\forall x)(\neg P(x)\lor((\forall y)((\exists z)(\neg Q(z,y))\lor(\exists t)(\neg R(y,t)))\\
=&(\forall x\forall y\exists z\exists t)(\neg P(x)\lor\neg Q(z,y)\lor\neg R(y,t))\\
=&\neg P(x)\lor\neg Q(g(x,y),y)\lor\neg R(y,g(x,y))
\end{aligned}\end{equation}
\subsection{}
\begin{equation}\begin{aligned}
G=&\neg(\forall xP(x))\lor((\exists x\forall s\forall t)(Q(x,t)\lor R(x,s,t)))\\
=&(\exists r\neg P(r))\lor((\exists x\forall s\forall t)(Q(x,t)\lor R(x,s,t)))\\
=&\exists r\exists x\forall s\forall t(\neg P(r)\lor Q(x,t)\lor R(x,s,t))\\
=&\neg P(f(s,t))\lor Q(g(s,t),t)\lor R(g(s,t),s,t)
\end{aligned}\end{equation}
\section{}
推理过程比较简单因此直接采用简单的因果推理
$$\because Read(Liming),\forall x Read(x)\rightarrow Smart(x)$$
$$\therefore Smart(Liming)$$
$$\because Smart(Liming)\land \neg Poor(Liming),\forall x \neg Poor(x)\land Smart(x)\rightarrow Happy(x)$$
$$\therefore Happy(Liming)$$
$$\because \forall x Happy(x)\rightarrow Exciting(x)$$
$$\therefore Exciting(Liming)$$
得证李明过着激动人心的生活
\end{document}