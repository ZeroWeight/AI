\documentclass[UTF8,a4paper]{ctexart}
\usepackage[utf8]{inputenc}
\usepackage{amsmath}
\usepackage{pdfpages}
\usepackage{graphicx}
\usepackage{wrapfig}
\usepackage{listings}
\usepackage{xcolor}
\usepackage{multicol}
\lstset{
    numbers=left, 
    numberstyle= \tiny, 
    keywordstyle= \color{ blue!70},
    commentstyle= \color{red!50!green!50!blue!50}, 
    frame=shadowbox, % 阴影效果
    rulesepcolor= \color{ red!20!green!20!blue!20} ,
    escapeinside=``, % 英文分号中可写入中文
    xleftmargin=2em,xrightmargin=2em, aboveskip=1em,
    framexleftmargin=2em
} 
\title{人工智能导论第一次大作业}
\author{张蔚桐\ 2015011493\ 自55}
\begin {document}
\maketitle
\section{程序编译与运行}
程序在Windows10 + Visual Studio 2015环境下采用C++,利用Qt5.7.0,OpenCV3,Eigen3开源库进行开发。如需要编译此程序,请在以上环境或兼容环境中编译。

请在64位Windows10系统内运行本程序

\begin {figure}
\includegraphics [width=\textwidth]{step1.png}
\caption{开始界面图}
\label{step1}
\includegraphics [width=\textwidth]{queryopen.png}
\caption{选择处理的文件}
\label{query}
\end {figure}
\begin {figure}
\includegraphics [width=\textwidth]{templateopen.png}
\caption{选择模板文件}
\label{template}
\includegraphics [width=\textwidth]{outputopen.png}
\caption{选择输出文件目录}
\label{output}
\end {figure}
\begin {figure}
\includegraphics [width=\textwidth]{templateopen.png}
\caption{选择模板文件}
\label{template}
\includegraphics [width=\textwidth]{outputopen.png}
\caption{选择输出文件目录}
\label{output}
\end {figure}
\begin {figure}
\includegraphics [width=\textwidth]{step2.png}
\caption{文件选定后界面}
\label{step2}
\includegraphics [width=\textwidth]{step3.png}
\caption{程序运行界面}
\label{step3}
\end {figure}
\begin {figure}
\includegraphics [width=\textwidth]{step2.png}
\caption{文件选定后界面}
\label{step2}
\includegraphics [width=\textwidth]{ans.png}
\caption{输出文件}
\label{output}
\end {figure}
如图\ref{step1}是程序的运行界面。单击左侧query files一行中右侧"..."按钮打开文件选择对话框,如图\ref{query}所示。选择需要的处理文件

需要注意的是,源文件必须是txt格式,同时,受到开发时间的限制,程序没有对文件是否合法进行进一步的检查,因此请使用和给定样例相同格式(32*32)的文件,否则可能导致未知的后果

同时,如图\ref{query}所示,程序允许选择多个待处理的文件,点击open打开这些文件,回到程序运行主界面

单击左侧template files一行中右侧"..."按钮打开文件选择对话框,如图\ref{template}所示。选择需要的模板

需要注意的是,源文件必须是txt格式,和前面处理文件不同的是,程序不允许选择多个模板文件,也就是说,程序仅能就一个模板进行多次的匹配操作。同样,点击open打开这些文件

单击左侧output directory一行中右侧"..."按钮打开文件选择对话框,如图\ref{output}所示。选择需要的输出文件目录

需要注意的是,在这里只能选定一个文件目录,输出的文件按照作业中给出的ground truth的格式,将以"output\_i\_j.txt"的名称输出到目标目录下。当然,可以不选择输出文件目录,这时,程序将不进行结果输出

回到程序运行界面发现要求的三个路径已经填写好,如图\ref{step2}所示,此时点击reset按键将清空选定的信息,点击OK开始执行程序

程序执行的过程中,会在右侧显示刚刚执行完成的图像的信息,同时下方进度条将显示整个批量执行的过程。当进度条为100\%时程序执行完毕,输出文件在选定的目录下,如图\ref{output}所示

整个程序执行过程中需要注意的一点是,程序并不会对路径信息和文件的合法性进行检查,因此建议采用上述方式选择对应的文件夹和文件,不要修改左侧显示文件路径信息的文本框。除非保证输入的路径完全正确,否则会导致未定义的行为
\section{搜索算法简述}
程序首先对图片中的白色部分进行连通域检测,划分出图片中的每一块白色连通域,这一部分由openCV中的函数和BFS衍生的函数完成

其次,程序对每个检测出的白色连通域单独检测。将这块连通域单独放入一个32*32的黑色图像中,对黑色部分进行连通域检测,如果发现全部的黑色部分均是一个联通域,说明白色连通域不构成闭合曲线,不符合题意,舍去,如果黑色连通域没有包含全图中的黑色区域,则说明白色部分是一个闭合曲线,保留之。这部分算法由BFS衍生而来,同时,为了处理有些图像中白色曲线出现“裂缝”情况,对白色图像在辨识之前进行模糊化处理。(保留的白色图像是未经模糊化的图像)

之后对检测剩余的图像进行剪枝。剪枝算法由简单的迭代完成,检查图片中的所有白色像素点,如果出现了周围白色像素点小于3个的白色像素点,我们认为这个点是一个分支的端点,将其染为黑色。逐次迭代直到没有这样的点

经过这些搜索,图像的F1-score已经能达到80\%。下面进行对线性变换的搜索提高准确度

利用搜索得到的图像中白色点的像素,解线性方程得到变换矩阵,验证变换矩阵的作用效果,将作用效果较好的矩阵得到的变换点留下,经过这样,F1-score已经能达到90\%

至此,输出并保存文件,核心算法结束。核心算法利用了程序中给定的“闭合多边形”和“线性变换”的条件
\section{其他工作量}
除了完成核心算法外,同时设计了用户界面,提供比较友好的交互形式。同时用户界面支持批量处理,并提供进度指示。利用OpenGL完成了图形的绘制工作,使得程序对用户比较友好

同时,在思路上,除了应用基本的图像搜索算法外,利用了题目中给出的“仿射…"等条件,在数据规模被减小到一定程度之后对模板和目标图像之间的线性变换进行搜索,进一步(约10\%)提升了准确度,但是,这种查找计算量相当大,因此只有利用搜索算法将查找集缩小到不超过40个的情况下才可以使用这种算法。本程序中,搜索算法和这种算法同时作用,提升了算法性能同时减少了运行时间

\end{document}